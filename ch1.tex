% \item Ingeniería de software
%   \begin{enumerate}
%   \item Definición 
%   \item Recurso humano
%   \item Recurso de software
%   \item Etapas del proceso de desarrollo de software
%   \item Ciclo de vida de desarrollo
%   \item Gestión del proyecto
%   \item Aseguramiento de calidad
%   \item Tiempos de lanzamiento
%   \item Estimación de tiempos y metas
%   \end{enumerate}
\chapter{Ingeniería de software}
\section{Definición}
La actividad de desarollo de software se relativamente nueva en el campo de la ciencia actual,
el crecimento y la importancia del software han originado la creación de nuevos campos y disciplinas 
en constante formación. Entre estas algunas han llegado a madurar de tal forma que ya forman parte de toda un área de la ciencia computacional como es el caso de la Ingeniería de software. La ingeniería de software es una disciplina de ingeniería que se interesa por todos los
aspectos de la producción de software, desde las primeras etapas de la especificación del
sistema hasta el mantenimiento del sistema después de que se pone en operación\cite{9786073206037}
